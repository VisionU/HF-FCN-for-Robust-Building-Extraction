\section{Introduction}
Extracting building rooftops automatically from very-high-resolution (VHR) satellite image has been an intensive research topic in the fields of remote sensing, compute vision  and computer graphics  over the last few decades. That's because building rooftops plays a critical role in  a diverse range of applications, such as  urban monitoring, disaster assessment, 3D city modelling, and military reconnaissance. Though a series of approaches have been proposed to solve this problem, it is still a big challenge to  
build a genetic and robust building detection system. This is mainly due to the following two reasons. One is that a large amount of buildings are occluded by shadows stem from itself or other buildings, it causes a significant difficulty for image segmentation. The other is that building rooftops possess diverse shapes and colors, therefore, they can be easily confused with similar objects such as cars, roads, and courtyards. 
	
	In previous literatures, large amounts of efforts achieve good performance in detecting buildings with special color, shapes, or textures. One popular way of extracting buildings is employing their shape information. It is observed that rooftops have more regular shapes, which usually are rectangular or combinations of several rectangles. A dozen years ago, Noronha and Nevatia \cite{noronha2001detection} designed a system that detects and constructs 3D models for rectilinear buildings from multiple aerial images. Firstly, hypotheses for rectangular rooftop were generated by grouping lines, then they were verified by searching for presence of predicted walls and shadows. Nosrati and Saeedi \cite{nosrati2009novel} pointed out that polygonal rooftops correspond to closed loops in a graph which represents the relationship between intersections of a pair of edge in an efficient way. In \cite{izadi2012three}, Izadi and Saeedi exploited a graph-based search to establish a set of rooftop hypotheses through examining the relationships of lines and line intersections. Cui \textit{et al.} \cite{cui2012complex} extracted buildings through the Hough transform (HT), but HT has notable drawbacks in parameter tuning and time complexity. Cote and Saeedi \cite{cote2013automatic} generated rooftop outline from selected corners in  multiple color and color-invariance spaces, further refined to fit the best possible boundaries through level-set curve evolution. In \cite{wang2015efficient}, Wang \textit{et al.}  presented a graph search-based perceptual grouping approach to hierarchically group line segments detected by EDLines \cite{akinlar2011edlines} into candidate rectangular buildings, computation complexity of the approach was reduced dramatically compared to \cite{noronha2001detection} \cite{izadi2012three} \cite{cote2013automatic} \cite{mayunga2007semi}. However, geometric primitives based methods suffer from three serious shortcomings. Firstly, they lack the ability of detecting arbitrarily shaped building rooftop. Secondly, they fail to  extract credible geometric features in buildings with inhomogeneous color distribution or low contrast with surroundings. Thirdly, it is hardly possible to process large-scale because of their high computational-complexity.	
	
	Several studies reported that buildings are often composed of homogeneous regions with similar color or texture nearby shadows in remote sensing images. Spectral features is a distinctive feature for object detection, for instance, shadows are commonly dark grey or black, vegetations are usually green or yellow with particular textures, and main roads are dim gray with different road marks in most case. According to these prior knowledge mentioned above, Ghaffarian \textit{et al.}  \cite{ghaffarian2014automaticPFICA} proposed an purposive fast independent component analysis (PFastICA) technique to separate building area from remote sensing image. However, Ghaffarian's approach fails to detect the buildings with significantly different coloured rooftops. In \cite{ghaffarian2014automaticsupervised}, illumination direction and shadow area information of training samples were collected firstly, and then a improved parallelepiped classification method was applied to classify the image pixels into building and non-building areas. In \cite{chen2014shadow}, Chen \textit{et al.} proposed a supervised building detection framework. At first, source image was divided into super-pixels using the SLIC \cite{achanta2012slic} algorithm, then shadow patches are recognized using LDA color feature and the SVM classifier. The rough segmentation of buildings is employed by an adaptive regional growth algorithm that considers the spatial relationship between shadows and buildings. Finally, buildings are segmented accurately using a level set model. Dornaika \textit{et al.} \cite{dornaika2015object} proposed a similar framework,
Firstly, remote sensing image is segmentation  by statistical region merging (SRM) algorithm , hybrid descriptor composed by color histograms and local binary patterns is used to represent each segmented region. Finally, each region was classified using machine learning tools and a gallery of training descriptors. However, a major problem of these methods is that they have no capability to detect building rooftop with significantly varying illumination in different parts.	
	
	The other studies presented synthetic approaches with fusion of shadow, spectral, and structure to extract building profiles from aerial images. Baluyan \textit{et al.} \cite{baluyan2013novel} proposed a method combining spectral and spatial features. Firstly, image is segmented into a set of rooftop candidates. Secondly, a SVM classifier is trained to distinguish rooftop regions or nonrooftop regions using extracted features in dataset. Finally, "histogram method" is devised to detect missed rooftops in previous step. In \cite{ngoautomatic}, Ngo \textit{et al.} presented a novel approach for automated detection of rectangular buildings. At the first step, image is decomposed into small homogeneous regions as candidates. At the second step, a merging process is then performed over regions having similar spectral traits to produce rectilinear building region in accordance with position of shadows. Li \textit{et al.} \cite{li2015robust} proposed a higher order conditional random field (CRF) based method, which incorporates both pixel-and segment-level information for the segmentation of rooftops. They claimed that the proposed model outperforms the state-of-the-art methods experiment at rooftops with complex structures and sizes.
	
   In our context, we aim at detecting polygonal buildings with different colors and  non-uniform illumination. Firstly, optical remote sensing image is segmented into a series of homogeneous regions with similar size by applying state-of-the-art over -segmentation algorithm, namely, superpixels extracted via energy-driven sampling (SEEDS)\cite{van2012seeds}. Secondly, all superpixels are represented by hybrid descriptor combining spectral and texture features, then expectation maximization (EM) is used to classify them into ten classes, shadow and vegetation are separated from these classes according to spectral information, rooftop hypothesises are separated from remaining classes. In other words, we re-partition source image into four clusters, namely, shadow, vegetation, probable buildings and others. Thirdly, we employ multi-label graph cut algorithm to refine previous results. The novelty of our algorithm is that it is not only more fast than most algorithms, but also makes good performance without training data.
   
   
   
   
	
	

